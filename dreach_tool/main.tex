\documentclass{llncs}
\usepackage{amsmath, amssymb, fancyvrb, multirow, color, stmaryrd}
\usepackage{svg}
\usepackage{wrapfig}
\usepackage{graphicx}
\usepackage{fancyvrb}
\usepackage{hyperref}
\usepackage{framed}
\usepackage{amsmath,amssymb,mathtools,nccmath}
\usepackage{paralist}
\usepackage{tabularx,booktabs}
\usepackage{multirow}
\usepackage{url}
\usepackage{tikz}
\usepackage[]{algorithm2e}

\newcommand{\enforce}{\mathsf{enforce}}
\newcommand{\traj}{\mathsf{traj}}
\newcommand{\drh}{\textsf{drh}}
\newcommand{\dReal}{\textsf{dReal}}
\newcommand{\dReach}{\textsf{dReach}}

\newcommand{\ite}[3]{\mathbf{if}\ #1\ \mathbf{then}\ #2\ \mathbf{else}\ #3\ \mathbf{fi}}
\newcommand{\power}[1]{2^{#1}}
\newcommand{\monus}{\;\,\mathtt{monus}\;\,}
\newcommand{\MA}{\mathcal{M}\!\!\mathcal{A}}
\newcommand{\AP}{\mathit{A\!P}}
\newcommand{\abs}{\ensuremath{\mathrm{abs}}}
\newcommand{\sign}{\ensuremath{\mathrm{sign}}}
\newcommand{\val}{\ensuremath{\mathit{val}}}
\newcommand{\eoe}{\hfill$\blacksquare$}

\input{preamble}

\title{\mbox{\dReach{}: $\delta$-Reachability Analysis for Hybrid Systems}}

\begin{document}


\mainmatter  % start of an individual contribution

\author{Soonho Kong, Sicun Gao, Wei Chen, and Edmund Clarke}
\authorrunning{S. Kong, S. Gao, W. Chen,  E. Clarke}
\institute{Computer Science Department, Carnegie Mellon University, USA}
\maketitle

\begin{abstract}
  \dReach{} is a bounded reachability analysis tool for nonlinear
  hybrid systems. It encodes reachability problems of hybrid systems
  to first-order formulas over real numbers, which are solved by
  delta-decision procedures in the SMT solver \dReal{}. In this way,
  \dReach{} is able to handle a wide range of highly nonlinear hybrid
  systems. It has scaled well on various realistic models
  from biomedical and robotics applications.
\end{abstract}

% Tool demonstration papers focus on the usage aspects of tools. As with
% regular tool papers, authors are strongly encouraged to make their
% tools publicly available, preferably on the web. Theoretical
% foundations and experimental evaluation are not required, however, a
% motivation as to why the tool is interesting and significant should be
% provided. Tool demonstration papers can have a maximum of 6 pages.
% They should have an appendix of up to 6 additional pages with details
% on the actual demonstration.

\input{intro.tex}
\input{background.tex}
\input{system.tex}
\input{encoding.tex}
\input{using.tex}
\input{exp.tex}
\input{conclusion.tex}
\bibliographystyle{abbrv}
\bibliography{tau_tacas}

%\newpage
%\section*{Appendix}
%\input{appendix.tex}

\end{document}

%%% Local Variables:
%%% mode: latex
%%% TeX-master: "main.tex"
%%% End:
