\documentclass{llncs}
\usepackage{amsmath, amssymb, fancyvrb, multirow, color, stmaryrd}
\usepackage{svg}
\usepackage{wrapfig}
\usepackage{graphicx}
\usepackage{fancyvrb}
\usepackage{hyperref}
\usepackage{framed}
\usepackage{amsmath,amssymb,mathtools,nccmath}
\usepackage{paralist}
\usepackage{tabularx,booktabs}
\usepackage{multirow}
\usepackage{url}
\usepackage{tikz}
\usepackage[]{algorithm2e}

\newcommand{\enforce}{\mathsf{enforce}}
\newcommand{\traj}{\mathsf{traj}}
\newcommand{\drh}{\textsf{drh}}
\newcommand{\dReal}{\textsf{dReal}}
\newcommand{\dReach}{\textsf{dReach}}

\newcommand{\ite}[3]{\mathbf{if}\ #1\ \mathbf{then}\ #2\ \mathbf{else}\ #3\ \mathbf{fi}}
\newcommand{\power}[1]{2^{#1}}
\newcommand{\monus}{\;\,\mathtt{monus}\;\,}
\newcommand{\MA}{\mathcal{M}\!\!\mathcal{A}}
\newcommand{\AP}{\mathit{A\!P}}
\newcommand{\abs}{\ensuremath{\mathrm{abs}}}
\newcommand{\sign}{\ensuremath{\mathrm{sign}}}
\newcommand{\val}{\ensuremath{\mathit{val}}}
\newcommand{\eoe}{\hfill$\blacksquare$}

\input{preamble}

\title{\mbox{\dReach{}: $\delta$-Reachability Analysis for Hybrid Systems}}

\begin{document}


\mainmatter  % start of an individual contribution

\author{Soonho Kong, Sicun Gao, Wei Chen, and Edmund Clarke}
\authorrunning{S. Kong, S. Gao, W. Chen,  E. Clarke}
\institute{Computer Science Department, Carnegie Mellon University, USA}
\maketitle

\begin{abstract}
  \dReach{} is a bounded reachability analysis tool for nonlinear
  hybrid systems. It encodes reachability problems of hybrid systems
  to first-order formulas over real numbers, which are solved by
  delta-decision procedures in the SMT solver \dReal{}. In this way,
  \dReach{} is able to handle a wide range of highly nonlinear hybrid
  systems. It has scaled well on various realistic models
  from biomedical and robotics applications.
\end{abstract}

% Tool demonstration papers focus on the usage aspects of tools. As with
% regular tool papers, authors are strongly encouraged to make their
% tools publicly available, preferably on the web. Theoretical
% foundations and experimental evaluation are not required, however, a
% motivation as to why the tool is interesting and significant should be
% provided. Tool demonstration papers can have a maximum of 6 pages.
% They should have an appendix of up to 6 additional pages with details
% on the actual demonstration.

\section{Introduction}\label{sec:intro}

% Need a paragrapgh or two to explain why the tool is interesting and
% significant should be provided.

\dReach{} is a bounded reachability analysis tool for hybrid systems.
It encodes bounded reachability problems of hybrid systems as
first-order formulas over the real numbers and solves them using
$\delta$-decision procedures in the SMT solver
\dReal{}~\cite{DBLP:conf/cade/GaoKC13}. \dReach{} is able to handle a
wide range of highly nonlinear hybrid
systems~\cite{CMSB14,DBLP:conf/fmcad/GaoKC13,DBLP:conf/hybrid/KapinskiDSA14,6868816}.
Figure~\ref{fig:prostate-example} highlights some of its features: on
the left is an example of some nonlinear dynamics that \dReach{} can
handle and on the right a visualized counterexample generated by
\dReach{} on this model.
\begin{figure}[!h]
  \subfloat[An example of nonlinear hybrid system model: off-treatment
  mode of the prostate cancer treatment model~\cite{CMSB14}\label{subfig-1:prostate}]{
    \includegraphics[width=0.45\textwidth]{images/prostatebw-mode2.pdf}
  }
  \hfill
  \subfloat[Visualization of a generated counterexample. Change in the shade of colors represents discrete mode changes.]{%
    \includegraphics[width=0.48\textwidth]{images/prostate}
  }
  \caption{An example of nonlinear dynamics and counterexample-generation.}
  \label{fig:prostate-example}
\end{figure}

It is well-known that the standard bounded reachability problems for
simple hybrid systems are already highly
undecidable~\cite{DBLP:conf/hybrid/AlurCHH92}.
Instead, we work in the framework of $\delta$-reachability of hybrid systems~\cite{DBLP:journals/corr/GaoKCC14}.
Here $\delta$ is an arbitrary positive rational number, provided by the user to
specify the bound on numerical errors that can be tolerated in the analysis.
For a hybrid system $H$ and an unsafe region $\unsafe$ (both encoded as logic formulas),
the $\delta$-reachability problem asks for one of the following answers:
\begin{itemize}
        \item {\sf safe}: $H$ cannot reach $\unsafe$.
        \item {\sf $\delta$-unsafe}: $H^{\delta}$ can reach $\unsafe^{\delta}$.
\end{itemize}
Here, $H^{\delta}$ and $\unsafe^{\delta}$ encode ($\delta$-bounded)
overapproximations of $H$ and $\unsafe$, defined explicitly as their
syntactic
variants. % (See Section~\ref{sec:delta-reachability} in the Appendix.)
It is important to note that the definition makes the answers no
weaker than standard reachability: when {\sf safe} is the answer, we
know for certain that $H$ does not reach the unsafe region (no
approximation is involved); when {\sf $\delta$-unsafe} is the answer,
we know that there exists some $\delta$-bounded perturbation of the
system that can render it unsafe.  Since $\delta$ can be chosen to be
arbitrarily small, {\sf$\delta$-unsafe} answers in fact discover
robustness problem in the system, which should be regarded as unsafe
indeed. We have proved that bounded $\delta$-reachabilty is decidable
for a wide range of nonlinear hybrid systems, even with reasonable
complexity bounds~\cite{DBLP:journals/corr/GaoKCC14}. This framework
provides the formal correctness guarantees of \dReach{}.

Apart from solving $\delta$-reachability, the following key features
of \dReach{} distinguish it from other existing tools in this
domain~\cite{DBLP:journals/jlp/FranzleTE10,DBLP:conf/cav/FrehseGDCRLRGDM11,DBLP:journals/tac/AlthoffK14,DBLP:conf/hybrid/Frehse05,DBLP:conf/icons/HerdeEFT08,DBLP:conf/rtss/ChenAS12,DBLP:conf/aaai/CimattiMT12}.
%insert explanations for each item.
\begin{enumerate}
\item Expressiveness: \dReach{} allows users to describe hybrid
  systems using first-order logic formulas over real numbers with a
  wide range of nonlinear functions. This allows users to specify the
  continuous flows using highly nonlinear differential equations, and
  the jump and reset conditions with complex Boolean combinations of
  nonlinear constraints. \dReach{} also faithfully translates mode
  invariants into $\exists\forall$ logic formulas, which can be
  directly solved under certain restrictions on the invariants.
\item Property-guided search: \dReach{} maintains logical encodings
  whose size is linear in the size of the inputs, of the reachable
  states of a hybrid system~\cite{DBLP:journals/corr/GaoKCC14}. The
  tool searches for concrete counterexamples to falsify the
  reachability properties, instead of overapproximating the full
  reachable states. This avoids the usual state explosion problem in
  reachable set computation, because the full set of states does not
  need to be explicitly stored. This change is analogous to the
  difference between SAT-based model
  checking~\cite{Biere:1999:SMC:646483.691738} and BDD-based symbolic
  model checking~\cite{McMillan:1993:SMC:530225}.
\item Tight integration of symbolic reasoning and numerical solving:
  \dReach{} delegates the reasoning on discrete mode changes to SAT
  solvers, and uses numerical constraint solving to handle nonlinear
  dynamics. As a result, it can combine the full power of both
  symbolic reasoning and numerical analysis algorithms. In particular,
  the existing tools for reachable set computation such as
  flow*~\cite{DBLP:conf/cav/ChenAS13} and
  SpaceEx~\cite{DBLP:conf/cav/FrehseGDCRLRGDM11} can be easily
  plugged-in as engines for solving the continuous part of the
  dynamics, while logic reasoning tools can overcome the difficulty in
  handling complex mode transitions.
\end{enumerate}

The paper is structured as follows. We provide background knowledge on
hybrid automata and SMT in Section~\ref{sec:prelim}. We describe the
system architecture of \dReach{} in Section~\ref{sec:system}, and give
details about the logical encoding in the tool in
Section~\ref{sec:encoding}. We explain the input format and
usage of the tool in Section 5. We report the experimental results in Section~\ref{sec:exp}.
%
% More details and examples are given in the Appendix.

%Realistic hybrid systems involves nonlinear ODEs with transcendental
%functions. \dReach{} allows users to specify a hybrid system in a
%nonlinear signature as it is without linearizing or overapproximating
%it. Users can provide the tool with a numerical error bound $\delta$,
%a bounded time horizon $[0, T]$, and a maximum number of mode switches
%$k$ for the analysis. As a result of analysis, \dReach{} will return
%either \textbf{$\delta$-sat} with a concrete counterexample, or
%\textbf{unsat} which does not involve numerical errors. We also
%provide a visualization for the $\delta$-sat case to help
%understand the analysis result.

% TODO: Need to differentiate this paper from FMCAD paper
%  - FMCAD: underlying solving techniques for SMT with ODEs
%  - TACAS: tool, encoding, using solver...

%%% Local Variables:
%%% mode: latex
%%% TeX-master: "main"
%%% End:

\input{background.tex}
\section{System Description}\label{sec:system}
The system architecture of \dReach{} is given in
Figure~\ref{sec:system}. It consists of an bounded model-checking
module and an SMT solver, \dReal{}. In the first phase, the Encoder
module translates an input hybrid system and specifications into a
logic formula. In the second phase, an SMT solver, \dReal{}, solves
the encoded $\delta$-reachability problem using a solving framework
that combines DPLL(T), Interval Constraint Propagation, and reliable
(interval-based) numerical integration.

\begin{figure}[h]
  \centering
  \includegraphics[width=\textwidth]{images/dreach_archi}
  \caption{Architecture of \dReach{}}\label{fig:system-description}
\end{figure}

We ask users to provide an input file with two parameters:
\begin{itemize}
\item An input file specifies a hybrid system, reachability properties
  in question, and time bounds on the continuous flow in each mode.
  The grammar is described in Section~\ref{sec:input-format}.
\item A bound on the number of mode changes.
\item A numerical error bound $\delta$.%, as explained in Section A5. in the Appendix.
\end{itemize}

From these inputs, \dReach{} generates a logical encoding that
involves existential quantification and universal quantification on
the time variables. The logical encoding is compact, always linear in
the size of the inputs. The tool then makes iterative calls to the
underlying solver \dReal{}~\cite{DBLP:conf/cade/GaoKC13} to decide the
reachability properties. When the answer is {\sf $\delta$-reachable},
\dReach{} generates a counterexample and its visualization. When the
answer is {\sf unreachable}, no numerical error is involved and a
logical proof of unsatisfiability can be provided~\cite{SYNASC14}.

%%% Local Variables:
%%% mode: latex
%%% TeX-master: "main"
%%% End:

\input{encoding.tex}
\section{Using \dReach{}}\label{sec:using-dreach}
% We now describe the input format and command line options of
% \dReach{}.
\subsection{Input Format}\label{sec:input-format}
The input format for describing hybrid systems and reachability properties consists of five
sections: macro definitions, variable declarations, mode definitions,
and initial condition, and goals. We focus on intuitive explanations here.
%, and the formal grammar is given in the Appendix.
Figure~\ref{fig:bouncing-ball-drh} shows how to describe a small
example hybrid system, an inelastic bouncing ball with air resistance.

\begin{itemize}
\item In macro definitions, we allows users to define macros in
  C-preprocessor style which can be used in the following
  sections. Macro expansions occur before the other parts are
  processed.

\item A variable declaration specifies a real variable and its domain
  in a real interval. \dReach{} requires special declaration for
  \textit{time} variable, to specify the upperbound of time duration.

\item A mode definition consists of mode id, mode invariant, flow, and
  jump.  \textit{id} is a unique positive integer assigned to a
  mode. An invariant is a conjunction of logic formulas which must hold
  in a mode. A flow describes the continuous dynamics of a mode by
  providing a set of ODEs. The first formula of \textit{jump} is
  interpreted as a guard, a logic formula specifying a condition to
  make a transition. Note that this allows a transition but does not
  force it. The second argument of \textit{jump}, $n$ denotes the
  target mode-id. The last one is \textit{reset}, a logic formula
  connecting the old and new values for the transition.

\item \textit{initial-condition} specifies the initial mode of a hybrid
system and its initial configuration. \textit{goal} shares the same
syntactic structure of \textit{initial-condition}.
\end{itemize}
\vspace{-1.0em}
\begin{figure}
  \centering
  \begin{Verbatim}[fontfamily=courier, frame=single, framesep=1mm, fontsize=\scriptsize]
 1   #define D 0.45
 2   #define K 0.9
 3   [0, 15] x; [9.8] g; [-18, 18] v; [0, 3] time;
 4   {   mode 1;
 5       invt: (v <= 0);  (x >= 0);
 6       flow: d/dt[x] = v; d/dt[v] = -g - (D * v ^ 2);
 7       jump: (x = 0) ==> @2 (and (x' = x) (v' = - K * v)); }
 8   {   mode 2;
 9       invt: (v >= 0); (x >= 0);
10       flow: d/dt[x] = v; d/dt[v] = -g + (D * v ^ 2);
11       jump: (v = 0) ==> @1 (and (x' = x) (v' = v)); }
12   init: @1 (and (x >= 5) (v = 0));
13   goal: @1 (and (x >= 0.45));
\end{Verbatim}
\caption{An example of \drh{} format: Inelastic bouncing ball with air
  resistance. Lines 1 and 2 define a drag coefficient $D = 0.45$ and
  an elastic coefficient $K = 0.9$. Line 3 declares variables
  $x, g, v,$ and $time$. At lines 4 - 7 and 8 - 11, we define two
  modes -- the falling and the bouncing-back modes respectively. At
  line 12, we specify the hybrid system to start at mode 1
  (\texttt{@1}) with initial condition satisfying
  $x \ge 5 \land v = 0$. At line 13, it asks whether we can have
  a trajectory ending at mode 1 (\texttt{@1}) while the height of the
  ball is higher than $0.45$.}
\label{fig:bouncing-ball-drh}
\end{figure}
\vspace{-1.6em}
\subsection{Command Line Options}
\dReach{} follows the standard Unix command-line usage:
\begin{Verbatim}[fontfamily=courier, framesep=1mm, fontsize=\small]
dReach <options> <drh file>
\end{Verbatim}
It has the following options:
\begin{itemize}
\item If \texttt{-k <N>} is used, set the unrolling bound $k$ as $N$
  (Default: 3). It also provides \texttt{-u <N>} and \texttt{-l <N>}
  options to specify upper- and lower-bounds of unrolling bound.
\item If \texttt{--precision <p>} is used, use precision $p$ (Default: $0.001$).
\item If \texttt{--visualize} is set, \dReach{} generates extra visualization data.
\end{itemize}
We have a web-based visualization toolkit which processes the
generated visualization data and shows the counterexample
trajectory. It provides a way to navigate and zoom-in/out trajectories
which helps understand and debug the target hybrid system better.

%%% Local Variables:
%%% mode: latex
%%% TeX-master: "main"
%%% End:

\section{Experimental Results}\label{sec:exp}

All benchmarks and data shown here are also available on the tool
website\footnote{\url{http://dreal.github.io}}. All experiments were
conducted on a machine with a 3.4GHz octa-core Intel Core i7-2600
processor and 16GB RAM, running 64-bit Ubuntu
12.04LTS. Table~\ref{tbl:exp} is a summary of the running time of the
tool on various hybrid system models.

\begin{table}[h]
  \centering
  \begin{tabular}{l|r|r|r|r|r|r|r|r}
    \hline
    \hline
    Benchmark    & \#Mode& \#Depth & \#ODEs & \#Vars  & Delta  & Result       & Time(s) & Trace \\
    \hline
    \hline
      AF-GOOD & 4     & 3        & 20     & 53      & 0.001     & SAT &  0.425    & 793K     \\
       AF-BAD & 4     & 3        & 20     & 53      & 0.001     & UNSAT &  0.074    & ---      \\
  AF-TO1-GOOD & 4     & 3        & 24     & 62      & 0.001     & SAT &  2.750    & 224K     \\
   AF-TO1-BAD & 4     & 3        & 24     & 62      & 0.001     & UNSAT &  5.189    & ---     \\
  AF-TO2-GOOD & 4     & 3        & 24     & 62      & 0.005     & SAT &  3.876    & 553K     \\
   AF-TO2-BAD & 4     & 3        & 24     & 62      & 0.001     & UNSAT &  8.857    & ---     \\
 AF-TSO1-TSO2 & 4     & 3        & 24     & 62      & 0.001     & UNSAT &  0.027    & ---     \\
       AF8-K7 & 8     & 7        & 40     & 101     & 0.001     & SAT & 10.478   & 3.8M      \\
      AF8-K23 & 8     & 23       & 40     & 293     & 0.001     & SAT & 135.29   & 11M      \\
    \hline
    \hline
    EO-K2  & 3     & 2        & 18     & 48      & 0.01    & SAT & 3.144    & 1.9M      \\
    EO-K11 & 3     & 11       & 99     & 174     & 0.01    & UNSAT & 0.969    & ---       \\
    \hline
    \hline
    QUAD-K1  & 2   & 1          & 34     & 89      & 0.01      & SAT & 2.386 &  10M \\
    QUAD-K2  & 2   & 2          & 34     & 125     & 0.01      & SAT & 4.971 &  13M \\
    QUAD-K3  & 4   & 3          & 68     & 161     & 0.01      & SAT & 13.755 & 42M \\
    QUAD-K3U & 4   & 3          & 68     & 161     & 0.01      & UNSAT & 2.846 & --- \\
    \hline
    \hline
    CT       & 2   & 2         & 10      & 41      & 0.005     & SAT & 345.84 & 3.1M\\
    CT       & 2   & 2         & 10      & 41      & 0.002     & SAT & 362.84 & 3.1M\\
    \hline
    \hline
    BB-K10 & 2     & 10       & 22     & 66      & 0.01        & SAT & 8.057     & 123K  \\
    BB-K20 & 2     & 20       & 42     & 126     & 0.01        & SAT & 39.196    & 171K  \\
    \hline
    \hline
  \end{tabular}
  \caption{
    \#Mode = Number of modes,
    \#Depth = Unrolling depth,
    \#ODEs = Number of ODEs in the unrolled formula,
    \#Vars = Number of variables in the unrolled formula,
    Trace = Size of the ODE trajectory,
    AF = Atrial Filbrillation,
    EO = Electronic Oscillator,
    QUAD = Quadcopter Control,
    CT = Cancer Treatment,
    BB = Bouncing Ball with Drag.
    % TIMES = Solving time in seconds, TO = Timeout (30min), PC = Proof
    % Checked, #PA = Number of proved axioms, #SP = Number of subproblems
    % generated by proof checking, TIMEPC = Proof-checking time in seconds, #D =
    % Number of iteration depth required in proof checking
}\label{tbl:exp}
\end{table}

\paragraph{Atrial Fibrillation.} We studied the Atrial Fibrillation
model as developed in~\cite{DBLP:conf/cav/GrosuBFGGSB11}.
The model
has four discrete control locations, four state variables, and
nonlinear ODEs. A typical set of ODEs in the model is:
\begin{eqnarray*}
\frac{du}{dt} &=& e + (u-\theta_v)(u_u-u ) v g_{fi} + wsg_{si}-g_{so}(u)\\
\frac{ds}{dt} &=& \displaystyle\frac{g_{s2}}{(1+\exp(-2k(u-us)))} -  g_{s2}s\\
\frac{dv}{dt} &=& -g_v^+\cdot v \hspace{1cm} \frac{dw}{dt} = -g_w^+\cdot w
\end{eqnarray*}
The exponential term on the right-hand side of the ODE is the sigmoid function, which often appears in modeling biological switches.
\paragraph{Prostate Cancer Treatment.} The Prostate Cancer Treatment
model~\cite{CMSB14} exhibits more nonlinear ODEs. The reachability
questions are
\begin{eqnarray*}
\frac{dx}{dt} &=& (\alpha_x
(k_1+(1-k_1)\frac{z}{z+k_2}-\beta_x( (1-k_3)\frac{z}{z+k_4}+k_3)) - m_1(1-\frac{z}{z_0}))x + c_1 x\\
\frac{dy}{dt} &=& m_1(1-\frac{z}{z_0})x+(\alpha_y (1- d\frac{z}{z_0}) - \beta_y)y+c_2y\\
\frac{dz}{dt} &=& \frac{-z}{\tau} + c_3z\\
\frac{dv}{dt} &=& (\alpha_x
(k_1+(1-k_1)\frac{z}{z+k_2}-\beta_x(k_3+(1-k_3)\frac{z}{z+k_4}))\\
& &- m_1(1-\frac{z}{z_0}))x + c_1 x + m_1(1-\frac{z}{z_0})x+(\alpha_y (1- d\frac{z}{z_0}) - \beta_y)y+c_2y
\end{eqnarray*}
\paragraph{Electronic Oscillator.} The EO model represents an electronic oscillator model that contains nonlinear ODEs such as the following:
\begin{eqnarray*}
\frac{dx}{dt} &=& - ax \cdot sin(\omega_1 \cdot \tau)\\
\frac{dy}{dt} &=& - ay \cdot sin( (\omega_1 + c_1) \cdot \tau) \cdot sin(\omega_2)\cdot 2\\
\frac{dz}{dt} &=& - az \cdot sin( (\omega_2 + c_2) \cdot \tau) \cdot cos(\omega_1)\cdot 2\\
\frac{\omega_1}{dt} &=& - c_3\cdot \omega_1\ \ \ \frac{\omega_2}{dt} = -c_4\cdot\omega_2\ \ \ \frac{d\tau}{dt} = 1
\end{eqnarray*}
\paragraph{Quadcopter Control.} We developed a model that contains the full dynamics of a quadcopter. We use the model to solve control problems by answering reachability questions. A typical set of the differential equations are the following:
\begin{eqnarray*}
\frac{\mathrm{d}\omega_x}{\mathrm{d}t} &=& L\cdot k\cdot (\omega_1^2 - \omega_3^2)(1/I_{xx})-(I_{yy} - I_{zz})\omega_y\omega_z/I_{xx}\\
\frac{\mathrm{d}\omega_y}{\mathrm{d}t} &=& L\cdot k\cdot(\omega_2^2 - \omega_4^2)(1/I_{yy})-(I_{zz} - I_{xx})\omega_x\omega_z/I_{yy}\\
\frac{\mathrm{d}\omega_z}{\mathrm{d}t} &=& b\cdot(\omega_1^2 - \omega_2^2 + \omega_3^2 - \omega_4^2)(1/I_{zz})-(I_{xx} - I_{yy})\omega_x\omega_y/I_{zz}\\
\frac{\mathrm{d}\phi}{\mathrm{d}t} &=& \omega_x + \displaystyle{\frac{\sin\left(\phi\right) \sin\left(\theta\right)}{{\left(\frac{\sin\left(\phi\right)^{2} \cos\left(\theta\right)}{\cos\left(\phi\right)} + \cos\left(\phi\right) \cos\left(\theta\right)\right)} \cos\left(\phi\right)}}\omega_y + \displaystyle\frac{\sin\left(\theta\right)}{\frac{\sin\left(\phi\right)^{2} \cos\left(\theta\right)}{\cos\left(\phi\right)} + \cos\left(\phi\right) \cos\left(\theta\right)}\omega_z\\
\frac{\mathrm{d}\theta}{\mathrm{d}t} &=& -(\displaystyle\frac{\sin\left(\phi\right)^{2} \cos\left(\theta\right)}{{\left(\frac{\sin\left(\phi\right)^{2} \cos\left(\theta\right)}{\cos\left(\phi\right)}\omega_y + \cos\left(\phi\right) \cos\left(\theta\right)\right)} \cos\left(\phi\right)^{2}} + \frac{1}{\cos\left(\phi\right)})\omega_y\\
& &\hspace{5cm}-\displaystyle\frac{\sin\left(\phi\right) \cos\left(\theta\right)}{{\left(\frac{\sin\left(\phi\right)^{2} \cos\left(\theta\right)}{\cos\left(\phi\right)} + \cos\left(\phi\right) \cos\left(\theta\right)\right)} \cos\left(\phi\right)}\omega_z \\
\frac{\mathrm{d}\psi}{\mathrm{d}t} &=& \displaystyle\frac{\sin\left(\phi\right)}{{\left(\frac{\sin\left(\phi\right)^{2} \cos\left(\theta\right)}{\cos\left(\phi\right)} + \cos\left(\phi\right) \cos\left(\theta\right)\right)} \cos\left(\phi\right)}\omega_y + \displaystyle\frac{1}{\frac{\sin\left(\phi\right)^{2} \cos\left(\theta\right)}{\cos\left(\phi\right)} + \cos\left(\phi\right) \cos\left(\theta\right)}\omega_z\\
\frac{\mathrm{d}{xp}}{\mathrm{d}t} &=& (1/m)(\sin(\theta)\sin(\psi)k(\omega_1^2 + \omega_2^2 +\omega_3^2+\omega_4^2) - k\cdot d\cdot{xp})\\
\frac{\mathrm{d}{yp}}{\mathrm{d}t} &=& (1/m)(-\cos(\psi)\sin(\theta)k(\omega_1^2 + \omega_2^2 +\omega_3^2+\omega_4^2) - k\cdot d\cdot{yp})\\
\frac{\mathrm{d}{zp}}{\mathrm{d}t} &=& (1/m)(-g-\cos(\theta)k(\omega_1^2 + \omega_2^2 +\omega_3^2+\omega_4^2) - k\cdot d\cdot{zp}\\
\frac{\mathrm{d}x}{\mathrm{d}t} &=& {xp}, \frac{\mathrm{d}y}{\mathrm{d}t} = {yp}, \frac{\mathrm{d}z}{\mathrm{d}t} = {zp}
\end{eqnarray*}


% \paragraph{Example encoding} The bounded reachability problem of a
% bouncing ball example (when $k = 3$) is encoded into the following
% shortened SMT2 formula.
% \begin{Verbatim}[fontfamily=courier, frame=single, framesep=1mm,  numbers=left, fontsize=\scriptsize]
% (set-logic QF_NRA_ODE)
% (declare-fun x_0_0 () Real) ...
% (declare-fun v_0_t () Real) ...
% (declare-fun time_0 () Real) ...
% (define-ode flow_1 ((= d/dt[x] v)
%                     (= d/dt[v] (+ (- 0.0 9.8) (* -0.45 (^ v 2.0))))))
% (define-ode flow_2 ((= d/dt[x] v)
%                     (= d/dt[v] (+ (- 0.0 9.8) (* +0.45 (^ v 2.0))))))
% (assert (<= 0.0 x_0_0)) ...
% (assert (<= v_10_t 18.0))
% (assert (<= 0.0 time_0))
% (assert (and (and (= v_0_0 0.0) (>= x_0_0 5.0)) (= mode_0 1.0) (=
% [x_0_t v_0_t] (integral 0. time_0 [x_0_0 v_0_0] flow_1)) (= mode_0
% 1.0) (forall_t 1.0 [0.0 time_0] (<= v_0_t 0.0)) (<= v_0_t 0.0) (<=
% ...
% x_9_t) (= [x_10_t v_10_t] (integral 0. time_10 [x_10_0 v_10_0]
% flow_1)) (= mode_10 1.0) (forall_t 1.0 [0.0 time_10] (<= v_10_t 0.0))
% (<= v_10_t 0.0) (<= v_10_0 0.0) (forall_t 1.0 [0.0 time_10] (>= x_10_t
% 0.0)) (>= x_10_t 0.0) (>= x_10_0 0.0) (= mode_10 1.0) (>= x_10_t
% 0.45))) (check-sat) (exit)
% \end{Verbatim}

%%% Local Variables:
%%% mode: latex
%%% TeX-master: "main"
%%% End:

\input{conclusion.tex}
\bibliographystyle{abbrv}
\bibliography{tau_tacas}

%\newpage
%\section*{Appendix}
%\input{appendix.tex}

\end{document}

%%% Local Variables:
%%% mode: latex
%%% TeX-master: "main.tex"
%%% End:
