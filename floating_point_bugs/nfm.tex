\documentclass{llncs}
\usepackage{amsmath, amssymb, fancyvrb, multirow, color, relsize, hyperref}

\newcommand{\rup}[1]{\ensuremath{\mathrm{#1}^{\uparrow}}}
\newcommand{\rdn}[1]{\ensuremath{\mathrm{#1}^{\downarrow}}}
\newcommand{\rzr}[1]{\ensuremath{\mathrm{#1}^{\to 0}}}
\newcommand{\rne}[1]{\ensuremath{\mathrm{#1}^{\cdot}}}
\newcommand{\dreal}{\textsf{dReal}}

\title{\mbox{Floating-point Bugs in Embedded GNU C Library}}

\author{Soonho Kong \and Sicun Gao \and Edmund M. Clarke}
\institute{Carnegie Mellon University, Pittsburgh, PA 15213}
\date{\today}


% \support{This research was sponsored by the National Science Foundation grants
% no. CNS1330014, no. CNS0926181 and no. CNS0931985, the GSRC under
% contract no. 1041377, the Semiconductor Research Corporation under
% contract no. 2005TJ1366, and the Office of Naval Research under award
% no. N000141010188.}

\begin{document}
\maketitle

\begin{abstract}
We report serious bugs in floating-point computations for evaluating elementary
functions in the Embedded GNU C Library. For instance, the sine
function can return values larger than $10^{53}$ in certain rounding
modes. Further investigation also exposed faulty implementations in
the most recent version of the library, which seemingly fixed some
bugs, but only by discarding user-specified rounding-mode
requirements. We discuss our experience in how these bugs were
spotted and how they affected the implementation process of our
SMT solver dReal.
\end{abstract}

\section{Introduction}\label{sec:intro}




We have found floating-point bugs in Linux systems using Embedded
GLIBC (EGLIBC) version 2.16 or older. EGLIBC is a variant of the GNU C
Library (GLIBC) which is used as the default implementation in many
distributions including Debian, Ubuntu, and their variants.

The following C program computes the value of $\sin(-2.437592)$ in
double-precision after setting the rounding direction to upward
($+\infty$).

\begin{Verbatim}[numbers=left, frame=single, fontsize=\relsize{-1}]
#include <math.h>
#include <fenv.h>
#include <stdio.h>

int main() {
    double x = -2.437592;
    fesetround(FE_UPWARD);
    printf("sin(%f)=%f\n", x, sin(x));
    return 0;
}
\end{Verbatim}

The IEEE754 standard~\cite{IEEE:1985:AIS} does not specify correct
rounding methods on elementary functions such as the exponential,
logarithm, and trigonometric functions. Programmers and engineers
usually expect the above program to print out an approximated value
around $\sin(-2.437592) \simeq -0.64727239229$ with an \emph{acceptable}
error bound. However, they all should agree that the result be in
the range between $-1.0$ and $1.0$, even in the worst case.

However, a surprising result appears if we compile and execute the
program in a machine using EGLIBC-2.16 or older (a typical example is
Ubuntu 12.04 LTS). The output is greater than $10^{53}$ which violates
the mathematical property of the sine function,
$\forall x. -1 \le \sin(x) \le 1$.

\begin{Verbatim}[frame=single, fontsize=\relsize{-1}]
$ gcc exp_bug.c -lm && ./a.out
sin(-2.437592)=191561981424936943059347927032148030287313979209416704.00
\end{Verbatim}
% $

Here is another C program computing $\cosh(3.113408)$ with directed
rounding toward $+\infty$. This example is more interesting because it
shows different results on Intel and AMD machines, and both of the
results have serious problems.

\begin{Verbatim}[numbers=left, frame=single, fontsize=\relsize{-1}]
#include <math.h>
#include <fenv.h>
#include <stdio.h>

int main() {
    double x = 3.113408;
    fesetround(FE_UPWARD);
    printf("cosh(%f) = %f\n", x, cosh(x));
    return 0;
}
\end{Verbatim}

In a machine with Intel Core i7 CPU, the program outputs
\textit{inf} while a machine with AMD Opteron processor produces
$-160.191709$.

\begin{Verbatim}[frame=single, fontsize=\relsize{-1}]
[INTEL] $ gcc cosh_bug.c -lm && ./a.out
cosh(3.113408) = inf
[AMD]   $ gcc cosh_bug.c -lm && ./a.out
cosh(3.113408) = -160.191709
\end{Verbatim}
% $

Note that $\cosh(3.113408) \simeq 11.2710174432$ and both results
\textit{inf} and $-160.191709$ are simply wrong. Moreover, each of the
wrong results has significant implications:

\begin{itemize}
\item Intel (\textit{inf}): It has a contagious effect in subsequent
  computations. \textit{inf} is a special value in the IEEE754
  standard which indicates an overflow in a computation. If one of
  subexpressions is evaluated to \textit{inf}, then in general the
  whole expression is also evaluated to either infinity ($+\infty$ or
  $-\infty$) or NaN (Not a Number).
\item AMD ($-160.191709$): Mathematically, $\cosh(x)$ is greater than or
  equal to $1$ for all real value $x \in \mathbb{R}$. As a result, programmers
  and engineers write algorithms based on the invariant $\forall x. \
  1 \le \cosh(x)$. This result, $-160.191709$, breaks the invariant
  and could cause an unexpected behavior in the algorithms.
\end{itemize}


\section{Testing of Math Functions in EGLIBC-2.16}\label{sec:bugs}

We have tested the implementations of the following math functions in EGLIBC-2.16:

\[
\mathtt{sin}, \mathtt{cos}, \mathtt{tan}, \mathtt{acos},
\mathtt{asin}, \mathtt{atan}, \mathtt{cosh}, \mathtt{sinh},
\mathtt{tanh}, \mathtt{exp}, \mathtt{log}, \mathtt{log10},
\mathtt{sqrt}.
\]

\begin{table}
  \centering
  \caption{Experiment Setup}
  \begin{tabular}{l|c|c||l|c|c}
    Function&               Domain&               Range& Function&    Domain &                    Range \\
    \hline
    \hline
    sin&    $[-10^{306},  10^{306}]$& $[-1.0,     1.0]$   & acos & $[-1.0,    1.0]$     & $[-\infty, +\infty]$\\
    cos&    $[-10^{306},  10^{306}]$& $[-1.0,     1.0]$   & asin & $[-1.0,    1.0]$     & $[-\infty, +\infty]$\\
    tan&    $[-10^{306},  10^{306}]$& $[-\infty, +\infty]$& atan & $[-1.0,    1.0]$     & $[-\infty, +\infty]$\\
    cosh&   $[-500, 500]$         & $[1.0,     +\infty]$& exp  & $[-100, 100]$        & $[0.0,     +\infty]$\\
    sinh&   $[-500, 500]$         & $[-\infty, +\infty]$& log  & $[10^{-306}, 10^{306}]$& $[-\infty, +\infty]$\\
    tanh&   $[-100, 100]$         & $[-1.0,     1.0]$   & log10& $[10^{-306}, 10^{306}]$& $[-\infty, +\infty]$\\
    sqrt&   $[0.0, 10^{306}]$      & $[0.0,     +\infty]$&      &                      &
  \end{tabular}
  \label{tbl:exp_setup}
\end{table}

For each function $f$, we sample $100,000$ random numbers from a
subset of function $f$'s domain. Table~\ref{tbl:exp_setup} shows each
function's sampling domain and range. We pick the sampling domain
carefully so that the result of the computations fit in a
double-precision variable. We consider the four rounding modes
supported by C99 standard~\cite{ISO:C99}:
\[
\cdot \text{ (nearest)}, \
\to\!\!0 \text{ (toward zero)}, \
\uparrow \text{ (toward $+\infty$)}, \
\downarrow \text{ (toward $-\infty$)}.
\]

\begin{table}[!h]
  \centering
  \caption{Experimental results on Intel and AMD machines:
    $\mathrm{f}^{\uparrow}$, $\mathrm{f}^{\downarrow}$, and
    $\mathrm{f}^{\to 0}$ indicate a function f with rounding mode toward
    $+\infty$, toward $-\infty$, and toward $0$ respectively.
    ``Inconsistent'' denotes the number of cases in which the difference
    of two results are larger than $2^{20}$ ULP (Unit of Least Precision).
    ``Incorrect'' denotes the number of cases in which
    $\mathrm{f}_{\mathrm{C}}(x)$ is out of $f$'s mathematical range.
    ``$\pm\infty$'' denotes the number of cases in which
    $\mathrm{f}_{\mathrm{C}}(x)$ is either $-\infty$ or $+\infty$.
  }
  \begin{tabular}{l||r|r|r|r||r|r|r|r}
\multirow{2}{*}{Function}&  \multicolumn{4}{c}{Intel}& \multicolumn{4}{|c}{AMD}\\
\cline{2-9}
       &   Inconsistent& Incorrect&$\pm\infty$& Total&  Inconsistent& Incorrect&$\pm\infty$& Total\\
    \hline\hline
    \rup{sin}&  10055&    446&   0&  10501&   9853&   450&   0&  10303\\
    \rdn{sin}&   9619&    497&   0&  10116&  10009&   450&   0&  10459\\
    \rzr{sin}&  10087&    436&   0&  10523&   9904&   423&   0&  10327\\
    \rup{cos}&  10097&    434&   0&  10531&   9880&   423&   0&  10303\\
    \rdn{cos}&   9815&    442&   0&  10257&   9910&   461&   0&  10371\\
    \rzr{cos}&   9737&    444&   0&  10181&   9913&   441&   0&  10354\\
    \rup{tan}&  12218&      0&   0&  12218&  12452&     0&   0&  12452\\
    \rdn{tan}&  12387&      0&   0&  12387&  12378&     0&   0&  12378\\
    \rzr{tan}&  12486&      0&   0&  12486&  12506&     0&   0&  12506\\
   \rup{cosh}&  18768&  30139& 935&  49842&  37091& 12295& 291&  49677\\
   \rdn{cosh}&  49766&      0&   0&  49766&  49673&     0&   0&  49673\\
   \rzr{cosh}&  49713&      0&   0&  49713&  49772&     0&   0&  49772\\
   \rup{sinh}&  47807&      0&   0&  47807&  47451&     0& 266&  47717\\
   \rdn{sinh}&  47493&      0&   0&  47493&  47676&     0&   0&  47676\\
   \rzr{sinh}&  47911&      0&   0&  47911&  48046&     0&   0&  48046\\
   \rup{tanh}&   3107&      0&   0&   3107&   3242&     0&   0&   3242\\
   \rdn{tanh}&   3135&      0&   0&   3135&   3268&     0&   0&   3268\\
    \rup{exp}&  47386&   2536&   0&  49922&  37883& 11708& 124&  49715\\
    \rdn{exp}&  49646&      0&   0&  49646&  49978&     0&   0&  49978\\
    \rzr{exp}&  50022&      0&   0&  50022&  49836&     0&   0&  49836
  \end{tabular}
  \label{tbl:exp_result}
\end{table}

For each sample $x$ and for each rounding mode $rnd$, we compute two
values $f^{\mathrm{rnd}}_{C}(x)$ and
$f^{\mathrm{rnd}}_{\mathrm{MPFR}}(x)$ where $f_{C}$ is a function $f$
in C standard library and $f_{\mathrm{MPFR}}$ is a function $f$ in the
GNU MPFR library~\cite{Fousse:2007:MMB:1236463.1236468}. MPFR supports
arbitrary-precision floating-point computation and we use it as a
reference implementation to have a comparison. The correctness of MPFR
is, of course, another issue. However, we do not discuss this issue
here. In the experiments, we use 256-bit precision for MPFR.

We have the following expectations for the two values
$f^{\mathrm{rnd}}_{C}(x)$ and $f^{\mathrm{rnd}}_{\mathrm{MPFR}}(x)$:
\begin{itemize}
\item Consistency: The difference between $f^{\mathrm{rnd}}_{C}(x)$
  and $f^{\mathrm{rnd}}_{\mathrm{MPFR}}(x)$ should not be too large.
  In this experiment, we set the threshold as $2^{20}$ ULP (Unit of
  Least Precision) which is the spacing between two adjacent
  floating-point numbers. Note that IEEE754 double-precision format
  has 53 bits of precision and $2^{20}$ ULP implies that it loses
  $20$-bit precision out of $53$. If $\lvert f^{\mathrm{rnd}}_{C}(x) -
  f^{\mathrm{rnd}}_{\mathrm{MPFR}}(x) \rvert > 2^{20} \mathrm{ULP}$,
  we label the case as ``inconsistent''.
\item Correctness: The value of $f^{\mathrm{rnd}}_{C}(x)$ should be in
  the range of the mathematical function $f$. For instance,
  $\sin^{\mathrm{rnd}}_{C}(x)$ has to be between -1.0 and 1.0 no matter
  how imprecise it is.
\end{itemize}

We run the experiments\footnote{Source code is available at
  \url{https://github.com/soonhokong/fp-test}} on two machines -- one
with Intel Core i7-2600 CPU (8-core, 3.40GHz) and another with AMD
Opteron Processor 6134 (32-core, 2.30GHz). Both of them are running
Ubuntu 12.04 LTS in which uses EGLIBC-2.15 for the C standard library
implementation. We use MPFR-3.1.1 and gcc-4.8.1 in the experiments.

The experimental results are summarized in table~\ref{tbl:exp_result}.
\begin{enumerate}
\item The implementations of $\mathrm{sin}$, $\mathrm{cos}$,
  $\mathrm{tan}$, $\mathrm{cosh}$, $\mathrm{sinh}$, $\mathrm{tanh}$
  and $\mathrm{exp}$ functions have severe problems when they are used
  with the non-default rounding modes (toward $\infty$, toward
  $-\infty$, and toward $0$). It is also not rare to have the
  problematic cases in practice. For example, $\rup{cosh}$ function
  gives wrong results almost 50\% of cases (49,842 out of 100,000).
\item We have not observed any problem under the default rounding mode
  (toward nearest representable number). Also the implementations of
  $\mathrm{acos}$, $\mathrm{asin}$, $\mathrm{atan}$, $\mathrm{tanh}$,
  $\mathrm{log}$, $\mathrm{log_{10}}$, and $\mathrm{sqrt}$ functions pass
  our tests.
\end{enumerate}

\section{Illustration of the Bug}



\begin{Verbatim}
Typical for plenty of basic numerical procedures:

- Break the domain into several compact pieces
- sin(x) breakpoints :

0, 2−26, 0.25, 0.855469, 2.426265, 105414350, ..., 21024, ...

- Local approximation algorithms in each piece • Bounded loops, mostly no pointers
- A lot of table lookup and bit-level hacks

In many cases, we can only reason with concrete semantics of bit-level operations.
• Tiny overapproximation can give false alarms.
• Undecidable theories over real numbers
• Mixed real/integers
• Large domain of inputs
• Huge proofs, if any


Using dReal, we found that the bug comes from an array
out-of-bound problem.
__sin(double x){ ...
/*---------------------------- 0.25<|x|< 0.855469---------------------- */
else if (k < 0x3feb6000) { ...
k=u.i[LOW_HALF]<<2; sn=(m>0)?__sincostab.x[k]:-__sincostab.x[k]; //table lookup ssn=(m>0)?__sincostab.x[k+1]:-__sincostab.x[k+1];
...
Here,   sincostab.x is an array of size 440.



\end{Verbatim}


\section{Fix in EGLIBC-2.17 and Remaining Problems}\label{sec:fix}

In EGLIBC-2.17, they provided a patch for the problem. The following
is a part of the new implementation of sine function (IEEE754
double-precision)\footnote{Available at
  \url{http://www.eglibc.org/cgi-bin/viewvc.cgi/branches/eglibc-2_17/libc/sysdeps/ieee754/dbl-64/s_sin.c?view=markup}}:

\begin{Verbatim}[numbers=left, frame=single, firstnumber=101,
  commandchars=\\\{\}, fontsize=\relsize{-1},
  label={\texttt{eglibc-2.17/libc/sysdeps/ieee754/dbl-64/s\_sin.c}},
  labelposition=topline ]
__sin(double x)\{
        double xx,res,t,cor,y,s,c,sn,ssn,cs,ccs,xn,a,da,db,eps,xn1,xn2;
#if 0
        double w[2];
#endif
        mynumber u,v;
        int4 k,m,n;
#if 0
        int4 nn;
#endif
        double retval = 0;

        \textbf{SET_RESTORE_ROUND_53BIT (FE_TONEAREST);}
\end{Verbatim}

At line 113, it resets the rounding mode to ``round to nearest'' and
compute the value of $\sin(x)$ while ignoring the user-specified
rounding mode. We found that the patch does not \emph{fix} the problem
correctly. For example, there is a case in which the value of
$\rup{\sin}_{C}(x)$ is smaller than the value of
$\rne{\sin}_{\mathrm{MPFR}}(x)$, which violates the semantics of
``toward $+\infty$'' rounding mode:
\begin{align*}
            \rup{\sin}_{C}(-3.93799) & = 0.7148414480838297\underline{66879659928236}\\
   \rne{\sin}_{\mathrm{MPFR}}(-3.93799) & = 0.7148414480838297\underline{71665831705916}
\end{align*}

\section{Our Experience with the Bugs}\label{sec:experience}
We found the bugs in the Embedded GNU C Library while investigating a
problem in our SMT solver, dReal~\cite{DBLP:conf/cade/GaoKC13}. dReal
implements a $\delta$-complete decision procedure for nonlinear
arithmetic, and should never return an {\sf unsat} answer on a
satisfiable formula. However, we obtained wrong answers on several
simple satisfiable formulas. During debugging, we concluded that the
only place that can go wrong is with interval computation in the ICP
(Interval Constraint Propagation) engine,
Realpaver~\cite{Granvilliers:2006:ARI:1132973.1132980}. Realpaver uses
the C standard library functions to change rounding modes and perform
interval computations. For instance, the following is the interval
implementation of $\mathrm{sinh}$ function in \texttt{realpaver-1.0}.

\begin{Verbatim}[frame=single, fontsize=\relsize{-1},
  label={\texttt{realpaver-1.0/src/rp\_interval.c}}]
void rp_interval_sinh(rp_interval result, rp_interval i)
{
  RP_ROUND_DOWNWARD();
  rp_binf(result) = sinh(rp_binf(i));
  RP_ROUND_UPWARD();
  rp_bsup(result) = sinh(rp_bsup(i));
}
\end{Verbatim}

For an input interval $i = [l, u]$, this implementation computes
$\mathrm{result} = [\rdn{sinh}(l), \rup{sinh}(u)]$ using math
functions in C standard library. Combined with the problem of
\texttt{eglibc-2.16}, this implementation caused unexpected behaviors
when the program is executed in machines running latest Ubuntu/Debian
OS. To fix the problem, we change the interval computation in
Realpaver using FILIB++~\cite{Lerch:2006:FFI:1141885.1141893}, a more
reliable interval computation library. However, we remark that unless
a floating point library is completely verified, there is always the
possibility for obtaining wrong answers. The user should always
require dReal to produce a proof of unsatisfiability for {\sf unsat}
answers, and validate it using the proof
checker~\cite{DBLP:conf/cade/GaoKC13}.
%Papers~\cite{Sankaranarayanan:2013:SAP:2499370.2462179} cite Realpaver

\section{Conclusion}\label{sec:conclusion}
We report serious bugs in floating-point computations for evaluating
elementary functions in the Embedded GNU C Library. It is not a
negligible numerical error but either a significant error ($2^{20}$
ULP) or mathematically incorrect result (i.e. $\sin(x) > 10^{53}$)
which can trigger severe problems in the following computations.
Moreover, the chances of having these results are not rare at all as
we have shown in section~\ref{sec:bugs}. The current fix does not
mitigate the problem but hides it.

\bibliographystyle{plain}
\bibliography{soon}

\end{document}

%%% Local Variables:
%%% mode: latex
%%% TeX-master: "nfm.tex"
%%% End:
